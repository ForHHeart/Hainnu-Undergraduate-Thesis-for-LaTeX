\documentclass[11pt,a4paper]{article}
%------------------------------------------------------------------------------------------%
%常规
\usepackage[top=1in,bottom=1in,left=1.1in,right=1in]{geometry}%设置页边距
\usepackage{setspace}%设置行间距
%\onehalfspacing
\usepackage{indentfirst}%设置首行缩进2格
\setlength{\parindent}{2em}
\usepackage{cite}%引用参考文献
\usepackage{fancyhdr}%设置页眉
\usepackage{palatino}
%------------------------------------------------------------------------------------------%
%数学
\usepackage{amsmath,amssymb,theorem,amscd,epic,curves,array}
\usepackage{pifont}%\ding{192}数字圈1、\ding{193}数字圈2
%------------------------------------------------------------------------------------------%
%表格
\usepackage{multirow,makecell,caption,float,rotating}%表格合并、垂直居中、标题、紧跟文字、旋转
\usepackage{booktabs}%三线表
%------------------------------------------------------------------------------------------%
%图片
\usepackage{graphicx,subfigure,wrapfig}%常规插图、子图、图文混排
\usepackage[all]{xy}
%------------------------------------------------------------------------------------------%
%中文
\usepackage{ctex,xeCJK,CJK}%中文宏包
\newcommand{\song}{\CJKfamily{song}}    % 宋体   (Windows自带simsun.ttf)
\newcommand{\fs}{\CJKfamily{fs}}        % 仿宋体 (Windows自带simfs.ttf)
\newcommand{\kai}{\CJKfamily{kai}}      % 楷体   (Windows自带simkai.ttf)
\newcommand{\hei}{\CJKfamily{hei}}      % 黑体   (Windows自带simhei.ttf)
\newcommand{\li}{\CJKfamily{li}}        % 隶书   (Windows自带simli.ttf)
%------------------------------------------------------------------------------------------%
%字体伪加粗
\usepackage{fontspec}
\setmainfont[AutoFakeBold]{Times New Roman}
\setCJKmainfont[AutoFakeBold]{SimSun}
%------------------------------------------------------------------------------------------%
%自定义
%目录
%\titlecontents{section}[0pt]{\bfseries\zihao{-4}\heiti}{\contentspush{\thecontentslabel.~}}{}{~\titlerule*[0.6pc]{$.$}~\contentspage}
%\titlecontents{subsection}[2em]{\zihao{-4}}{\thecontentslabel}{}{~\titlerule*[0.6pc]{$.$}~\contentspage}
%\titlecontents{subsubsection}[4em]{\zihao{-4}}{\thecontentslabel}{}{~\titlerule*[0.6pc]{$.$}~\contentspage}
%\titlecontents{chapter}[0pt]{\bfseries \zihao{-4}\heiti}{\contentspush{第\thecontentslabel 章\quad}}{}{~\titlerule*[0.6pc]{$.$}~\contentspage}[\vspace{12pt}]
%小标题
\usepackage{titlesec}
\usepackage{titletoc}
\titleformat*{\section}{\zihao{3}\bfseries\song}
\titleformat*{\subsection}{\zihao{-3}\bfseries\song}
\titleformat*{\subsubsection}{\zihao{4}\bfseries\song}
%------------------------------------------------------------------------------------------%
%------------------------------------------------------------------------------------------%
%------------------------------------------------------------------------------------------%


\begin{document}
%------------------------------------------------------------------------------------------%
%------------------------------------------------------------------------------------------%
%封面
\newpage
\thispagestyle{empty}
\linespread{1.6}{
	\mbox{}\\
	\vskip8mm
	\begin{figure}[htbp]
		\centering
		\includegraphics[width=12.15cm,height=6.99cm]{logo.jpg}
	\end{figure}
	
	\begin{center}
		\Huge{\songti \pmb {\pmb{本~~科~~生~~毕~~业~~论~~文}}}\\
		\vspace{2cm}
		\LARGE\heiti{论文题目:\underline {这是论文题目}}%输入题目
		
		\vspace{2.1cm}
		
		\begin{flushleft}
			\Large\songti{
				\hspace{4.2cm}姓~~~~~~名: \underline{~~~~~~~~~~~~某某某~~~~~~~~~~~~}%输入姓名
				\\
				\hspace{4.2cm}学~~~~~~号: \underline{~~~~~~000000000000~~~~~~}% 输入学号
				\\
				\hspace{4.2cm}专~~~~~~业: \underline{~~~~~数学与应用数学~~~~}%输入专业
				\\
				\hspace{4.2cm}年~~~~~~级: \underline{~~~~~~~~~~~~2018 级~~~~~~~~~~~~}%输入年级
				\\
				\hspace{4.2cm}学~~~~~~院: \underline{~~~~~数学与统计学院~~~~}%输入院系
				\\
				\hspace{4.2cm}完成日期: \underline{~~2022~年~4~月~20~日~~}%输入日期
				\\
				\hspace{4.2cm}指导教师: \underline{~~~~~~某某某~~教授~~~~~~~}}% 输入指导教师
%------------------------------------------------------------------------------------------%
%------------------------------------------------------------------------------------------%
%独创性声明
\newpage\pagenumbering{Roman}
\linespread{2.1}{
	\begin{center}
		\LARGE\heiti{本科生毕业论文独创性声明}
	\end{center}
	\vspace{28pt}
	
	\Large\fangsong{~~~~~~~~本人声明所呈交的毕业论文是本人在导师指导下进行的研究工作及取得的研究成果,
		除了文中特别加以标注和致谢的地方外,本论文中没有抄袭他人研究成果和伪造数据等行为 。与我一同工作的同志对本研究所做的任何贡献均已在论文中作了明确的说明并表示谢意。}
	
	\vspace*{1.2cm}
	\Large{~~~~~~~~论文作者签名: \underline{~~~~~~~~~~~~~~~~~~~~~}~~~~~~日期: \underline{~~~~~~~~~~~~~~~~~~~~~}}\\
	\vspace*{1.2cm}
	\begin{center}
		\LARGE\heiti{本科生毕业论文使用授权声明}
	\end{center}
	\vspace{28pt}
	\Large\fangsong{~~~~~~~~海南师范大学有权保留并向国家有关部门或机构送交毕业论文的复印件和磁盘,
		允许毕业论文被查阅和借阅。本人授权海南师范大学可以将本毕业论文的全部或部分内容编入有关数据库进行检索,可以采用影印、
		缩印或其他复印手段保存、汇编毕业论文。}
	
	\vspace*{1.2cm}\Large{~~~~~~~~论文作者签名: \underline{~~~~~~~~~~~~~~~~~~~~~}~~~~~~ 日期: \underline{~~~~~~~~~~~~~~~~~~~~~}
		
		\vspace*{0.8cm}
		~~~~~~~~指导教师签名: \underline{~~~~~~~~~~~~~~~~~~~~~}~~~~~~ 日期: \underline{~~~~~~~~~~~~~~~~~~~~~}}}
\end{flushleft}
\end{center}}				
%------------------------------------------------------------------------------------------%
%------------------------------------------------------------------------------------------%
%目录
\newpage
\linespread{1.1}{\def\contentsname{\centering\huge{\pmb{\kaishu{目~~~~录}}}}
\LARGE\tableofcontents}
%------------------------------------------------------------------------------------------%
%------------------------------------------------------------------------------------------%
%摘要
\newpage
\pagenumbering{arabic}
\linespread{1.8}
\title{\heiti 题目}%输入中文题目
\author{\fangsong 作者: XXX~~~指导教师: XXX~~教授\\
	海南师范大学数学与统计学院,海口,571158}
\date{}
\maketitle
\renewcommand\baselinestretch{1.4}\selectfont
\begin{quotation}{\heiti ~~摘要:}{\fangsong
		摘要正文
	}
	
	{\heiti ~关键词:}
	{\fangsong
		关键词1;关键词2;关键词3;关键词4;关键词5
	}\par
\end{quotation}	
	
\begin{center}
		\Large{\textbf {Topic} }%输入英文题目
		\\
		\large{ Author: XXX ~~~~Tutor: Professor~XXX  %输入者和指导教师(英文名)
			\\
			(College of Mathematics and Statistics, Hainan Normal University, Haikou, 571158)}
\end{center}
\renewcommand\baselinestretch{1.4}\selectfont
\begin{quotation} \textbf{Abstract:}
		the body of the abstract
		
	\textbf{Keywords:} Keyword1; Keyword2; Keyword3; Keyword4; Keyword5
\end{quotation}\par

%------------------------------------------------------------------------------------------%
%------------------------------------------------------------------------------------------%
%正文
\newpage
\zihao{-4}
\section{引言}

引言正文

%------------------------------------------------------------------------------------------%
%------------------------------------------------------------------------------------------%
%参考文献

\renewcommand\baselinestretch{1.1}\selectfont
\renewcommand\refname{\textbf{参考文献}}
\begin{thebibliography}{1}
	\fangsong
	\addcontentsline{toc}{section}{\textbf{参考文献}}
	
	\bibitem{paper1}参考文献正文
	
	\bibitem{paper2}参考文献正文
	
\end{thebibliography}

%------------------------------------------------------------------------------------------%
%------------------------------------------------------------------------------------------%
%致谢

\renewcommand\baselinestretch{1.4}\selectfont
\addcontentsline{toc}{section}{\textbf{致谢}}
\begin{flushleft}
	\zihao{3}\textbf{致谢}
\end{flushleft}

致谢正文

%------------------------------------------------------------------------------------------%
%------------------------------------------------------------------------------------------%
%附录

\addcontentsline{toc}{section}{\textbf{附录}}
\begin{flushleft}
	\zihao{3}\textbf{附录}
\end{flushleft}

附录正文

\end{document}
