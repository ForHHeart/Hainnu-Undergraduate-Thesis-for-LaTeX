%------------------------------------------------------------------------------------------%
%常见表格宏包
\usepackage{multirow,makecell,caption,float,rotating}%表格合并、垂直居中、标题、紧跟文字、旋转
\usepackage{booktabs}%三线表
%----------------------------------------------------------------%
%----------------------------------------------------------------%
%在线生成LaTeX表格网站
%适用于绝大多数常规表格,可从word、excel直接ctrl+v到网站生成所需表格,
%点击generate按钮,并使用copy to clipboard按钮复制到剪贴板,粘贴到tex文件即可使用
https://www.tablesgenerator.com/latex_tables
%----------------------------------------------------------------%
%----------------------------------------------------------------%
%1.三线表
%三线表要素
\toprule%表格顶线(加粗)
\midrule%表格中线
\bottomrule%表格底线(加粗)
%使用\toprule、\miderule、\bottomrule时会出现竖线的断层,如果要使用竖线通常使用\hline或\cline{}
%示例1
\begin{table}[H]
	\centering
	\setlength{\abovecaptionskip}{0cm}
	\setlength{\belowcaptionskip}{-0.2cm}
	\caption{数形结合在函数主线中的应用}
	\zihao{5}
	\begin{tabular*}{0.98\textwidth}{@{\extracolsep{\fill}}ll}
		\toprule
		\multicolumn{1}{c}{章节名} & \multicolumn{1}{c}{数形结合的应用}                                                              \\
		\midrule
		函数的基本性质                 & 根据函数图像研究函数的单调性、最值、奇偶性等性质                                                                 \\
		基本初等函数                  & \begin{tabular}[c]{@{}l@{}}1.结合函数图像研究指数函数的单调性和特殊点\\ 2. 结合函数图像研究对数函数的单调性和特殊点\end{tabular} \\
		函数的初步应用                 & 结合函数图像判断函数的零点与方程根的关系                                                                     \\
		三角函数 &
		\begin{tabular}[c]{@{}l@{}}1. 任意角的大小\\ 2. 利用单位圆中定义任意角的正弦、余弦、正切\\ 3.用三角函数线表示任意角的三角函数以及同角三角函数的基本关系\\ 4.利用三角函数线推到诱导公式\\ 5.由平移三角函数线和描点法绘制三角函数图像\\ 6.通过对的图像进行平移、伸缩变换得到图像\\ 7.由单位圆上的三角函数线和向量推导两角差的余弦公式\end{tabular} \\
		数列                      & 等差数列、等比数列的图象    \\
		\bottomrule                                                                       
	\end{tabular*}
\end{table}
%示例2
\begin{table}[H]
	\centering
	\footnotesize
	\begin{tabular}{c|c|cc|cc|cc|cc}
		\toprule
		\multirow{2}{*}{\textbf{Category}} &
		\multirow{2}{*}{\textbf{Models}} &
		\multicolumn{2}{c|}{\textbf{Enron}} &
		\multicolumn{2}{c|}{\textbf{UCI}} &
		\multicolumn{2}{c|}{\textbf{Youtube}} &
		\multicolumn{2}{c}{\textbf{TKY}} \\
		&            & \textbf{AUC}    & \textbf{MAP}    & \textbf{AUC}    & \textbf{MAP}    & \textbf{AUC}    & \textbf{MAP}    & \textbf{AUC}    & \textbf{MAP}    \\ \midrule
		\multirowcell{3}{\textbf{Heuristics methods}} &
		CN &	0.6858 &	0.6777 &	0.5688 &	0.6947 &	0.5696 &	0.5621 &	0.5037 &	0.504 \\
		& Newton       & 0.6847 & 0.6908 & 0.569  & 0.5662 & 0.5    & 0.5    & 0.5    & 0.5    \\
		\multirowcell{4.5}{\textbf{Static Link   Prediction}} & node2vec     & 0.5    & 0.5    & 0.5    & 0.5    & 0.5    & 0.5    & 0.5    & 0.5    \\ \midrule
		& GCN          & 0.5288 & 0.5346 & 0.5306 & 0.552  & 0.5031 & 0.5083 & 0.4532 & 0.4884 \\
		& GAT          & 0.5247 & 0.5327 & 0.4569 & 0.4835 & 0.5628 & 0.5199 & 0.4633 & 0.4928 \\ \midrule
		\multirow{3}{*}{\textbf{Dynamic Link   Prediction}} &
		DySAT &0.6304 &	0.617 &	0.6378 &	0.6109 &	0.6539 &	0.6369 &	0.6327 &	0.6169 \\
		& DynamicTriad & 0.5559 & 0.5479 & 0.6406 & 0.6274 & 0.6854 & 0.6757 & 0.569  & 0.5746 \\
		& GC-LSTM      & 0.5131 & 0.537  & 0.5653 & 0.5617 & 0.7067 & 0.7068 & 0.6163 & 0.6051 \\ \midrule
		\textbf{Ours}                                      & ModelName    & 0.7059 & 0.7125 & 0.7199 & 0.6947 & 0.7572 & 0.7242 & 0.7446 & 0.7996\\ \bottomrule
	\end{tabular}
\end{table}
%----------------------------------------------------------------%
%----------------------------------------------------------------%
%二、对表头高度进行调整以及控制表格的宽度
%直接使用\caption{}可能会导致表头与表格之间的距离太大,调用\setlength对表头高度进行调整
\setlength{\abovecaptionskip}{0cm}
\setlength{\belowcaptionskip}{-0.2cm}
%表格因为文字大小的关系,会导致论文中每个表格宽度不一,将\begin{tabular}换为\begin{tabular*},并调整参数即可实现宽度统一
		\begin{tabular*}{0.9\textwidth}{@{\extracolsep{\fill}}ll}
			%示例
			\begin{table}[H]
				\centering
				\setlength{\abovecaptionskip}{0cm}
				\setlength{\belowcaptionskip}{-0.2cm}
				\caption{数形结合在概率与统计主线中的应用}
				\zihao{5}
				\begin{tabular*}{0.98\textwidth}{@{\extracolsep{\fill}}ll}
					\hline
					\multicolumn{1}{c}{章节名} & \multicolumn{1}{c}{数形结合的应用}                                                                          \\
					\hline
					概率                      & \begin{tabular}[c]{@{}l@{}}1.用Venn 图表示事件的关系与交、并运算\\ 2.用树状图表示基本事件及其概率\\ 3.用几何关系计算基本事件的概率\end{tabular} \\
					统计                      & \begin{tabular}[c]{@{}l@{}}1.频率分布表、频率分布直方图、频率分布折线图、茎叶图\\ 2.由散点图分析变量的相关关系\end{tabular} \\
					\hline              
				\end{tabular*}
			\end{table}
			%----------------------------------------------------------------%
			%----------------------------------------------------------------%
			%三、在表格中插图
			%示例
			\begin{table}[H]
				\centering
				\setlength{\abovecaptionskip}{0cm}
				\setlength{\belowcaptionskip}{-0.2cm}
				\caption{直线与圆的位置关系}
				\zihao{5}
				\begin{tabular*}{0.98\textwidth}{@{\extracolsep{\fill}}ccc}
					\hline
					相离               & 相切         & 相交            \\ \hline
					方程组无解            & 方程组有且只有一个解 & 方程组有两个解       \\
					d>r & d=r        & d<r \\~
					&     ~       & ~              \\
					\begin{minipage}{0.1\textwidth}				
						\includegraphics[scale=0.7]{3.png}
					\end{minipage}&\begin{minipage}{0.1\textwidth}				
						\includegraphics[scale=0.7]{4.png}
					\end{minipage}            &\begin{minipage}{0.1\textwidth}				
						\includegraphics[scale=0.7]{5.png}
					\end{minipage}               \\~
					&    ~        &   ~            \\ \hline
				\end{tabular*}
			\end{table}
			%----------------------------------------------------------------%
			%----------------------------------------------------------------%
			%四、制作教案长表格
			%示例
			\begin{table}[H]
				\centering
				\footnotesize
				\begin{tabular}{|lll|}
					\hline
					\multicolumn{1}{|l|}{课题名称} & \multicolumn{2}{l|}{《二元一次方程组》} \\ \hline
					\multicolumn{3}{|l|}{一、教学目标}                                \\ \hline
					\multicolumn{3}{|l|}{(1)能根据题目所给的信息列出相应的方程或方程组;}             \\ 
					\multicolumn{3}{|l|}{(2)能列举或者判断一组数值是不是二元一次方程组的解;}           \\ 
					\multicolumn{3}{|l|}{\begin{tabular}[c]{@{}l@{}}(3)通过对我国古代数学著作中所提出的数学问题进行分析,感受我\\ 国古代数学家的智慧,提升家国情怀,提高对数学的学习兴趣。\end{tabular}}                            \\ \hline
					\multicolumn{3}{|l|}{二、教学重难点}                               \\ \hline
					\multicolumn{3}{|l|}{(1)能够列出二元一次方程或判断一组数是不是方程的解;}           \\
					\multicolumn{3}{|l|}{(2)理解二元一次方程组及其解的概念。}                   \\ \hline
					\multicolumn{3}{|l|}{三、教学过程}                                \\ \hline
					\multicolumn{1}{|l|}{(1)创设情境,导入新课} & \multicolumn{1}{l|}{\begin{tabular}[c]{@{}l@{}}~\\\begin{minipage}{0.1\textwidth}				
								\includegraphics[scale=0.15]{10.png}
							\end{minipage}\\ 
							~~~~~~~~师:请同学们看到这张图片,假设老牛背\\上有x个包裹,小马背上有,个包裹,根据左\\边图片中的信息我们可以得到什么方程?根据\\右边图片中的信息我们又能得到一个什么样的\\方程?\\
							~~~~~~~~生:$\left\{\begin{matrix}
								x-y=2\\x+1=2(y-1)				
							\end{matrix}\right.$
					\end{tabular}} & \begin{tabular}[c]{@{}l@{}}~~~~~~~~设计意\\图:在课堂引\\入中以顺应\\式的方式将\\我国《孙子算\\法》一文中的\\数学问题插\\入到课堂,让\\学生感受我\\国古代数学\\家获得的成\\果,滲透家国\\情怀教育。\end{tabular} \\ \hline				
				\end{tabular}
			\end{table}