%------------------------------------------------------------------------------------------%
%常见数学公式宏包
\usepackage{amsmath,amssymb,theorem,amscd,epic,curves,array}
\usepackage{pifont}%\ding{192}数字圈1、\ding{193}数字圈2
%----------------------------------------------------------------%
%----------------------------------------------------------------%
%在线生成LaTeX数学公式网站
%适用于绝大多数常规公式,实时生成公式
https://www.latexlive.com/
%----------------------------------------------------------------%
%----------------------------------------------------------------%
%推荐数学公式识别为LaTeX代码的免费截图软件:Mathpix Snip,可能需要挂梯子
%使用邮箱注册账号,下载客户端软件即可截图识别,识别度高,支持中英文,英文更准确
https://mathpix.com/
%----------------------------------------------------------------%
%----------------------------------------------------------------%
%对公式进行编号,两种最简单有效的方法
%方法一:自动编号\begin{equation}
%适合编号是一个字母,如果需要自动编子序号,可使用\tag \label \ref或者导言区自定义设置equation环境
\begin{equation}
    your formulas
\end{equation}
%方法二:手动编号\eqno(),一个笨而有效的方法
$$
your formulas\eqno(1.1)
$$
